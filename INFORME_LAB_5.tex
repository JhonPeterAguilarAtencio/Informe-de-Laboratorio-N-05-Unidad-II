% Generated by GrindEQ Word-to-LaTeX 
\documentclass{article} %%% use \documentstyle for old LaTeX compilers

\usepackage[english]{babel} %%% 'french', 'german', 'spanish', 'danish', etc.
\usepackage{amssymb}
\usepackage{amsmath}
\usepackage{txfonts}
\usepackage{mathdots}
\usepackage[classicReIm]{kpfonts}
\usepackage[dvips]{graphicx} %%% use 'pdftex' instead of 'dvips' for PDF output

% You can include more LaTeX packages here 


\begin{document}

%\selectlanguage{english} %%% remove comment delimiter ('%') and select language if required


\noindent 

\noindent 

\noindent 

\noindent 

\noindent \textbf{ p\'{a}g. }4\textbf{}

\noindent UNIVERSIDAD PRIVADA DE TACNA

\noindent \includegraphics*[width=3.24in, height=2.36in, keepaspectratio=false]{image1}

\noindent 

\noindent 

\noindent 

\noindent 

\noindent 

\noindent 

\noindent INGENIERIA DE SISTEMAS

\noindent 

\noindent 

\noindent 

\noindent TITULO:

\noindent 

\noindent \textbf{INFORME DE LABORATORIO N 05}

\noindent \textbf{CURSO:}

\noindent INTELIGENCIA DE NEGOCIOS

\noindent \textbf{DOCENTE (ING):}

\noindent Patrick Cuadros Quiroga

\noindent \textbf{INTEGRANTES:}

\noindent 

\noindent Aguilar Atencio, Jhon Peter     (2015053222)

\noindent 

\noindent 

\noindent 

\noindent Contenido

\noindent TAREA 1: IMPORTACION DE DATOS USANDO EL WIZARD 3TAREA 2: CREAMOS NUESTRO PRIMER PAQUETE DTSX 7

\noindent 

\noindent 

\noindent 

\noindent 

\noindent 

\noindent 

\noindent 

\noindent 

\noindent 

\noindent 

\noindent 

\noindent 

\noindent 

\noindent 

\noindent 

\noindent 

  

\noindent 

\noindent 

\noindent 

\noindent 

\noindent 

\noindent 
\section{TAREA 1: IMPORTACION DE DATOS USANDO EL WIZARD}

\noindent 

\begin{enumerate}
\item  Crear una base de datos -- BDTEST
\end{enumerate}

\noindent \includegraphics*[width=3.96in, height=3.15in, keepaspectratio=false, trim=0.00in 0.25in 0.00in 0.51in]{image2}

\begin{enumerate}
\item    Elegimos la base de datos origen\includegraphics*[width=3.95in, height=3.64in, keepaspectratio=false]{image3}

\item  Elegimos la base de datos destino
\end{enumerate}

\noindent \includegraphics*[width=3.17in, height=3.24in, keepaspectratio=false]{image4}

\begin{enumerate}
\item  \includegraphics*[width=3.93in, height=3.62in, keepaspectratio=false]{image5}Selecionamos la primera opcion

\item  Selecionamos las tablas Human.Deparment y Person.Address
\end{enumerate}

\noindent \includegraphics*[width=3.92in, height=3.64in, keepaspectratio=false]{image6}

\begin{enumerate}
\item  \includegraphics*[width=3.96in, height=3.66in, keepaspectratio=false]{image7}Activamos para que guarde el paquete
\end{enumerate}

\noindent 

\begin{enumerate}
\item  Asignamos nombre y ruta del archivo
\end{enumerate}

\noindent \includegraphics*[width=3.82in, height=3.43in, keepaspectratio=false, trim=0.04in 0.12in 0.09in 0.12in]{image8}

\begin{enumerate}
\item  \includegraphics*[width=3.70in, height=3.51in, keepaspectratio=false, trim=0.13in 0.08in 0.09in 0.01in]{image9}Presionamos en finalizar

\item  Una vez terminado el procedimiento , presionamos en el boton Cerrar
\end{enumerate}

\noindent \includegraphics*[width=3.96in, height=3.61in, keepaspectratio=false, trim=0.00in 0.00in 0.00in 0.06in]{image10}

\noindent 
\section{TAREA 2: CREAMOS NUESTRO PRIMER PAQUETE DTSX}

\begin{enumerate}
\item \textbf{\underbar{ }}Crear un proecto de Business Intelligense : Integration Services
\end{enumerate}

\noindent \includegraphics*[width=3.95in, height=2.47in, keepaspectratio=false]{image11}

\begin{enumerate}
\item  Agregamos el paquete Import01.dtsx generado anteriormente
\end{enumerate}

\noindent \includegraphics*[width=3.96in, height=2.61in, keepaspectratio=false]{image12}

\begin{enumerate}
\item  Configuramos una nueva conexion
\end{enumerate}

\noindent \includegraphics*[width=3.80in, height=3.23in, keepaspectratio=false, trim=0.04in 0.10in 0.05in 0.07in]{image13}

\begin{enumerate}
\item  Agregamos una Tarea de flujo de datos y una tarea de Ejecutar SQL
\end{enumerate}

\noindent \includegraphics*[width=3.95in, height=2.40in, keepaspectratio=false]{image14}

\begin{enumerate}
\item  Abrimos la Tarea de Ejecutar SQL y configuramos la consulta para obtener la cantidad de registros en SQL Statement
\end{enumerate}

\noindent \includegraphics*[width=3.72in, height=3.01in, keepaspectratio=false, trim=0.13in 0.09in 0.05in 0.24in]{image15}

\begin{enumerate}
\item  Asignar la variable en ``Numero de Registros Sales''
\end{enumerate}

\noindent \includegraphics*[width=3.83in, height=3.25in, keepaspectratio=false, trim=0.05in 0.06in 0.09in 0.08in]{image16}

\noindent 

\noindent 

\noindent 

\noindent 

\begin{enumerate}
\item  Finalmente Ejecutamos
\end{enumerate}

\noindent \includegraphics*[width=3.48in, height=2.59in, keepaspectratio=false, trim=0.15in 0.93in 2.57in 0.55in]{image17}

\noindent 

\noindent 


\end{document}

